% 设置为 Beamer 文档类型,设置字体为 10pt,长宽比为16:9,数学字体为 serif 风格
\documentclass[10pt,aspectratio=169,mathserif]{ctexbeamer}

%%%%-----导入宏包-----%%%%
\usepackage{style/zjubeamer}
\usepackage{xeCJK}
\usepackage{amsmath,amsfonts,amssymb,bm}
\usepackage{color}
\usepackage{graphicx,hyperref,url}
\usepackage{comment}
%%%%%%%%%%%%%%%%%%

%%%%-----设置字体-----%%%%
% \setsansfont{Helvetica}
% \setsansfont{Times New Roman}

% 设置 Beamer 主题
\beamertemplateballitem
\AtBeginSection[]
{
  \begin{frame}<beamer>
    \frametitle{\textbf{目录}}
    \textbf{\tableofcontents[currentsection]}
  \end{frame}
}

%%%%----首页信息设置----%%%%
\AtBeginDocument{%
\title[基于自监督学习的摸鱼神经网络]{\fontsize{13pt}{18pt}\selectfont {基于自监督学习的摸鱼神经网络}}
%%%%----标题设置
\subtitle{\fontsize{9pt}{14pt}\selectfont \textbf{SlackNet: How to Slack Off Happily via Self-Supervised Learning}}
%%%%----个人信息设置
\author[咸鱼]{%
  \begin{tabular}{ll}%
    答辩人: & 咸鱼 \tabularnewline%
    学号:   & 21700000 \tabularnewline%
    专业:   & 摸鱼技术与工程 \tabularnewline%
    导师:   & 老咸鱼 \tabularnewline%
  \end{tabular}
}
%%%%----机构信息
\institute[ZJU]{老和山职业技术学院}
%%%%----日期信息
\date[\today]{\today}
}

\begin{document}

% 标题页
\begin{frame}
\titlepage
\end{frame}

% 提纲页
\section*{目录}
\label{sec:toc}
\begin{comment}
PPT主要内容不需照着念,略一停留就可进入背景介绍。
\end{comment}
\begin{frame}
  \frametitle{\textbf{目录}}
  \textbf{\tableofcontents}
\end{frame}
% section* 目录 (end)

\section{研究背景}
\label{sec:background}
\begin{comment}
背景介绍尽量简练,2-3页为宜,但信息量要足。
目的是给出研究背景(对应选题意义)、现状,总结当前工作的不足,从而引出自己的工作。 
\end{comment}
\begin{frame}{摸鱼}
  摸鱼是人类生存的必要条件。
\end{frame}
% section 绪论 (end)

\section{研究思路}
\label{sec:idea}
\begin{comment}
全文工作思路,1-2页。理清逻辑,让观众到此明白问题轮廓和自己的工作全貌。
\end{comment}
\begin{frame}{思路}
  摸鱼必不可少。需要学会摸鱼。如何优雅地摸鱼。
\end{frame}
% section 研究思路 (end)

\section{研究内容}
\label{sec:content}
\begin{comment}
讲解自己的详细工作要突出思路和重点。
不一定在语言表达上涉及太多细节,比如,用过多公式讲解他人的工作步骤应避免,属于自己的工作要在视觉和语言上进行标注和区别。实验结果的表示要精炼,让人容易理解。
对比试验要公平,有说服力,对比对象要新,要有对比意义,从而体现自己的工作价值(这是研究方法和论文写作阶段都有的问题,但是答辩时常被质问)。
讲解包含可能的额外演示。
\end{comment}
\begin{frame}{网络设计}
端到端的摸鱼神经网络设计。
\end{frame}

\begin{frame}{实验结果}
本文提出的方法使得摸鱼效率与准确率提升了50\%。
\end{frame}
% section 研究内容 (end)

\section{总结与展望}
\label{sec:conclusion}
\begin{comment}
最后一定要有总结,突出个人工作和结果
展望和工作的不足之处不宜多,淡化处理。
致谢可以写,但要简练,并且不要照着读,一句话即可,如:“最后,感谢所有关心和帮助过我的每一个人,感谢各位专家和评委老师”。
\end{comment}
\begin{frame}{总结}
还是摸鱼适合老子。
\end{frame}

\begin{frame}{展望}
今后也要做一条快乐摸鱼的咸鱼。
\end{frame}
% section 总结与展望 (end)

\end{document}
